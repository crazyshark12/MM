\chapter*{Введение}							% Заголовок
\addcontentsline{toc}{chapter}{Введение}	% Добавляем его в оглавление
Обзор, введение в тему, обозначение места данной работы в мировых исследованиях и т.п.

Нефтепроводы — инженерно-технические сооружения трубопроводного транспорта, предназначенное для транспорта нефти.  Сооружение и обслуживание трубопровода весьма дорогостоящее, но тем не менее — это наиболее дешёвый способ транспортировки газа и нефти. По этой причине, удешевление процесса транспортировки нефти и повышение эффективности её передачи является важнейшей задачей при строительстве трубопроводов.

Из скважин вместе с нефтью поступают пластовая вода, попутный нефтяной газ (ПНГ), твердые частицы механических примесей (горных пород, затвердевшего цемента.), поэтому поступающую из скважин нефть и газ нужно очистить. В данной работе мы концентрируемся на проблеме оседания твердой фазы на дне трубы.

Широко признано, что неэффективное удаление твердых частиц представляет собой ключевую проблему для технической и экономической осуществимости транспортировочных работ. Отложение шлама в стволе трубопровода может привести к прихвату труб, существенно высокому сопротивлению и крутящему моменту, снижению объёмов поступающего вещества. Таким образом, точное прогнозирование оседания бурового шлама в трубе жизненно важно для надежной конструкции и правильного выполнения процесса транспортировки.

Проблемы гидромеханики иногда не имеют точного аналитического решения, а проведение эксперимента, получаение и обработка данных явлеттся довольно затратным в плане финансирования и времени. Численные методы решения, в отличие от вышеописанных, не используют ресурсы лабороторий, обходятся дешевле и могут быть довольно легко модифицированны. Именно поэтому для решения обозначенной проблемы мы использовали методы вычислительной гидродинамики. 

В данной работе использовался программный пакет для вычислительной гидродинамики OpenFoam. Этот продукт предоставляет широкий спектр функций для решения любых задач, от сложных потоков жидкости, включающих химические реакции, турбулентность и теплопередачу, до акустики, механики твердого тела и электромагнетизма. OpenFoam имеет открытый исходный код и бесплатную модель распространения, что является очень важным и акутальным преимуществом.


\textbf{Целью} данной работы является изучение распределения осадочного слоя при транспортировке нефти\ldots

Для~достижения поставленной цели необходимо было решить следующие задачи:
\begin{enumerate}
  \item Выбрать оптимальные физические модели.
  \item Научиться работать в программном пакете OpenFoam.
  \item Выбрать оптимальные математические и вычислительные модели.
  \item Смоделировать течение жидкости при разных углах наклона
\end{enumerate}

\textbf{Основные положения, выносимые на~защиту:}
\begin{enumerate}
  \item Первое положение
  \item Второе положение
  \item Третье положение
  \item Четвертое положение
\end{enumerate}

\textbf{Научная новизна:}
\begin{enumerate}
  \item Впервые \ldots
  \item Впервые \ldots
  \item Было выполнено оригинальное исследование \ldots
\end{enumerate}

\textbf{Научная и практическая значимость} \ldots

\textbf{Степень достоверности} полученных результатов обеспечивается \ldots Результаты находятся в соответствии с результатами, полученными другими авторами.

\textbf{Апробация работы.}
Основные результаты работы докладывались~на:
перечисление основных конференций, симпозиумов и т.п.

\textbf{Личный вклад.} Автор принимал активное участие \ldots

\textbf{Публикации.} Основные результаты по теме диссертации изложены в ХХ печатных изданиях~\cite{bib1,bib2,bib3,bib4,bib5},
Х из которых изданы в журналах, рекомендованных ВАК~\cite{bib1,bib2,bib3}, 
ХХ --- в тезисах докладов~\cite{bib4,bib5}.

\textbf{Объем и структура работы.} Диссертация состоит из~введения, четырех глав, заключения и~двух приложений. Полный объем диссертации составляет ХХХ~страница с~ХХ~рисунками и~ХХ~таблицами. Список литературы содержит ХХХ~наименований.

\clearpage